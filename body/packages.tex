\usepackage{array}
\newcommand{\ccr}[1]{\makecell{{\color{#1}\rule{1cm}{1cm}}}}

% 添加浮动环境
\usepackage{float}

% 使用颜色包
%\usepackage[dvipsnames]{xcolor}

% 用于添加代码的设置。
\usepackage{listings}
%\usepackage{ctex}
% 用来设置代码的样式
\lstset{
    basicstyle          =   \sffamily,          % 基本代码风格 % \ttfamily等
    breaklines=true, % 自动换行
    %keywordstyle        =   \bfseries,          % 关键字风格
    keywordstyle        =   {\bfseries\color{NavyBlue}},
    emphstyle={\bfseries\color{Rhodamine}}, %强调部分的风格
    alsoletter={.}, %把.也看做字母,这样在完整字符串匹配时有用         
    commentstyle        =   \rmfamily\itshape,  % 注释的风格,斜体
    stringstyle         =   \rmfamily,  % 字符串风格
    flexiblecolumns,                % 别问为什么,加上这个
    numbers             =   left,   % 行号的位置在左边
    %numbersep=2em,
    showspaces          =   false,  % 是否显示空格,显示了有点乱,所以不现实了
    numberstyle         =   \zihao{-6}\ttfamily,    % 行号的样式,小六号,tt等宽字体
    showstringspaces    =   false,
    captionpos          =   tl,      % 这段代码的名字所呈现的位置,t指的是top上面
    %frame               =   tb,   % 显示边框 %lrtb左右顶底
    frame = shadowbox, %框为阴影盒子样式 
    backgroundcolor={\color{yellow!20!white}},%背景颜色
    rulecolor = {\color{purple}},%边框颜色
    rulesepcolor = {\color{orange}},%shadowbox右边和下边的偏移部分的颜色
    rulesep=6pt,%边框偏移量
    %framexleftmargin     =   7mm,  % 代码边框向左延伸5mm,以包含行号
    xleftmargin=2em,xrightmargin=2em,aboveskip=1em,    % 设置左/右/上间距
    %escapeinside=``, %逃逸字符。用双`符号包裹的字符串可以直接输出。最直接的用途是用来在代码中添加中文(listings默认不支持中文)
}

\lstdefinestyle{Go}{
    language        =   Go, % 语言选Go
    basicstyle      =   \zihao{-5}\ttfamily,
    numberstyle     =   \zihao{-6}\ttfamily,
    keywordstyle    =   \color{blue},
    keywordstyle    =   [2] \color{teal},
    stringstyle     =   \color{magenta},
    commentstyle    =   \color{gray}\ttfamily,
    breaklines      =   true,   % 自动换行,建议不要写太长的行
    columns         =   fixed,  % 如果不加这一句,字间距就不固定,很丑,必须加
    basewidth       =   0.5em,
}